% **************************************************
% * Opciones de clase y globales
% **************************************************
\documentclass[%
    paper=A4,               % Tamaño de papel A4
    twoside,                % Páginas pares e impares con diseño diferenciado
    parskip=full,           % Renglón libre entre párrafos
    10pt,                   % Tamaño de fuente. No inferior a 10 puntos.
    titlepage=on,           % Página independiente para el título
    captions=tableabove,    % Los títulos de las tablas en la parte superior
]{article}                  % Basado en el tipo "article" 

% **************************************************
% * Idioma y paquete de formato
% **************************************************
\usepackage[spanish]{babel}
\usepackage{tfgteleco}

% **************************************************
% * Datos para la portada del TFG
% **************************************************
\title{Diseño y desarrollo de una funcionalidad de búsqueda y filtrado de cursos en el catálogo de cursos en línea}
\author{Aarón Riveiro Vilar}
\titores{Manuel Caeiro Rodríguez}
\curso{2025/2026}

% **************************************************
% * Cabeceras (opcionales)
% **************************************************
\cabeceiras{true}

% **************************************************
% * Inicio del documento
% **************************************************
\begin{document}

\maketitle
% **************************************************
% * Tábla de contenidos (opcional)
% **************************************************
\tableofcontents
\clearpage

% **************************************************
% * Introducción
% **************************************************
\section{Introducción}
En los últimos años, el crecimiento sostenido de los sistemas de información digitales ha dado lugar a repositorios con volúmenes de datos cada vez mayores y más heterogéneos. En este contexto, la mera disponibilidad de la información ya no resulta suficiente: es necesario dotar a los sistemas de mecanismos eficaces que permitan localizar, filtrar y presentar los datos de forma relevante para el usuario. La disciplina de la recuperación de información (Information Retrieval) estudia precisamente los métodos y modelos que permiten acceder de manera eficiente a grandes colecciones de documentos, maximizando la relevancia de los resultados ofrecidos frente a una consulta determinada \cite{information_retrieval}.

A medida que aumenta el tamaño y la complejidad de los repositorios, la experiencia de usuario pasa a depender en gran medida de la calidad del sistema de búsqueda y filtrado. No solo es importante la rapidez de respuesta, sino también la precisión y exhaustividad de los resultados, así como la capacidad del sistema para interpretar la intención de búsqueda del usuario y ofrecer mecanismos de refinamiento progresivo mediante filtros, facetas o criterios adicionales \cite{retrieval_accuracy}. Un sistema de búsqueda limitado a coincidencias literales sobre un único campo de texto puede resultar insuficiente en entornos donde los datos presentan múltiples dimensiones y atributos relevantes.

Por ello, el diseño e implementación de estrategias avanzadas de búsqueda y filtrado constituye un elemento clave en cualquier sistema de información orientado al usuario final, especialmente cuando se persigue facilitar el acceso eficiente a contenidos estructurados y mejorar la usabilidad global de la plataforma.

% **************************************************
% * Contexto
% **************************************************
\subsection{Contexto del proyecto}
El proyecto DACEM (Digitising Academic Catalogues for Enhanced Mobility) \cite{dacem}, aprobado en la convocatoria 2023 del programa Erasmus+ para Asociaciones Estratégicas en Educación Superior (código de proyecto 2023-1-ES01-KA220-HED-000160344), tiene como finalidad el desarrollo de una plataforma de catálogo de cursos dirigida a instituciones educativas europeas. Su objetivo principal es ofrecer un sistema que permita publicar y mantener información actualizada sobre titulaciones y asignaturas de forma accesible, estructurada e interoperable, facilitando así los procesos de movilidad académica internacional.

El sistema de información asociado a dicha plataforma se fundamenta en tecnologías de software libre y está concebido para poder desplegarse tanto en entornos locales como en infraestructuras en la nube. Más allá del almacenamiento estructurado de datos académicos, el correcto funcionamiento del sistema depende de su capacidad para gestionar, organizar y recuperar la información de forma eficiente. En este contexto, los mecanismos de búsqueda y filtrado desempeñan un papel esencial, ya que constituyen el principal medio de interacción entre el usuario y el repositorio de cursos. La calidad de estos mecanismos condiciona directamente la usabilidad del sistema, la precisión de los resultados obtenidos y la experiencia global de navegación.

% **************************************************
% * Motivación
% **************************************************
\subsection{Motivación}
Uno de los retos fundamentales en los sistemas de información que gestionan grandes volúmenes de datos estructurados no reside únicamente en su almacenamiento, sino en la forma en que estos datos son recuperados y presentados al usuario. A medida que un repositorio crece en número de elementos y en complejidad de atributos, los mecanismos de búsqueda simples basados en coincidencias textuales directas pueden resultar insuficientes para ofrecer resultados relevantes y ajustados a las necesidades reales de consulta.

Esta problemática es especialmente significativa en entornos donde los datos presentan múltiples dimensiones descriptivas —como titulaciones, asignaturas, áreas de conocimiento, idiomas, créditos o modalidades— y donde el usuario puede tener criterios de búsqueda diversos y combinados. La ausencia de mecanismos avanzados de filtrado y refinamiento progresivo limita la capacidad del sistema para adaptarse a diferentes perfiles de usuario y escenarios de consulta, reduciendo la eficiencia en la localización de información pertinente.

En el ámbito actual del desarrollo de aplicaciones web y sistemas de gestión de contenidos, las tendencias apuntan hacia la implementación de motores de búsqueda más sofisticados, capaces de indexar múltiples campos, ponderar resultados según relevancia, incorporar criterios de filtrado dinámico y mejorar la experiencia de usuario mediante interfaces intuitivas. Estas soluciones no solo optimizan el acceso a la información, sino que contribuyen a una mayor usabilidad y aprovechamiento del sistema en su conjunto.

El presente Trabajo Fin de Grado se enmarca en este contexto, proponiendo el diseño y desarrollo de una mejora en los mecanismos de búsqueda y filtrado del catálogo de cursos de la plataforma DACEM. El objetivo es evolucionar desde un sistema de coincidencia textual básica hacia una solución más flexible y precisa, alineada con las buenas prácticas actuales en recuperación de información y desarrollo web, empleando tecnologías abiertas e integradas en el entorno existente.

% **************************************************
% * Objetivos y requisitos
% **************************************************
\section{Objetivos y requisitos}
El presente trabajo se centra en la mejora de los mecanismos de búsqueda y filtrado del catálogo de cursos desarrollado en el marco del proyecto DACEM. A diferencia de otros desarrollos orientados a la recopilación o estructuración de datos, este proyecto aborda la fase de recuperación y presentación de la información, poniendo el foco en la interacción entre el usuario y el sistema.

El objetivo principal consiste en analizar las limitaciones del sistema actual de consulta e implementar una solución que permita mejorar la precisión, flexibilidad y usabilidad de los resultados obtenidos. Para ello, se definen a continuación los requisitos funcionales y no funcionales que deben guiar el diseño e implementación de la propuesta.

Se entiende por requisito funcional aquel que describe las capacidades concretas que debe ofrecer el sistema, mientras que los requisitos no funcionales establecen condiciones relativas a su comportamiento, rendimiento, integración y mantenibilidad, sin definir directamente nuevas funcionalidades.

% **************************************************
% * Requisitos funcionales
% **************************************************
\subsection{Requisitos funcionales}

% **************************************************
% * Sistema de búsqueda
% **************************************************
\subsubsection{Sistema de búsqueda}
\begin{enumerate}
    \item El sistema debe permitir la realización de búsquedas textuales sobre el catálogo de cursos a partir de una cadena de consulta introducida por el usuario.

    \item La búsqueda no debe limitarse exclusivamente al título del curso, sino que deberá contemplar múltiples campos relevantes del modelo de datos, tales como descripción, área de conocimiento, titulación asociada u otros atributos significativos que formen parte de la información estructurada del catálogo.

    \item El sistema debe soportar coincidencias parciales y no estrictamente literales, permitiendo la localización de resultados aun cuando la consulta no coincida exactamente con el texto almacenado. En este sentido, se deberá contemplar la normalización de mayúsculas y minúsculas y, en su caso, la gestión de caracteres acentuados.

    \item El sistema deberá ordenar los resultados de búsqueda según un criterio de relevancia, de manera que los elementos más pertinentes en relación con la consulta aparezcan en las primeras posiciones.

    \item El sistema debe garantizar que las búsquedas con consultas vacías o ambiguas no provoquen errores ni comportamientos inesperados, definiendo un comportamiento consistente en estos casos (por ejemplo, mostrando todos los resultados o solicitando una consulta válida).

    \item El sistema deberá integrarse con el modelo de datos existente sin alterar su estructura fundamental, reutilizando en la medida de lo posible los mecanismos de indexación y consulta proporcionados por la plataforma.

    \item El sistema debe permitir la extensión futura del mecanismo de búsqueda, de forma que puedan incorporarse nuevos campos o criterios sin requerir una reestructuración completa de la solución implementada.
\end{enumerate}

% **************************************************
% * Sistema de filtrado
% **************************************************
\subsubsection{Sistema de filtrado}

% **************************************************
% * Integración y rendimiento
% **************************************************
\subsubsection{Integración y rendimiento}

% **************************************************
% * Requisitos no funcionales
% **************************************************
\subsection{Requisitos no funcionales}

% ! TODO

% **************************************************
% * Diseño
% **************************************************
\section{Diseño}

% **************************************************
% * Desarrollo
% **************************************************
\section{Desarrollo}

% **************************************************
% * Conclusiones
% **************************************************
\section{Conclusiones}

% **************************************************
% * Bibliografía
% **************************************************
\bibliographystyle{IEEEtran}
\bibliography{references}
\clearpage

% **************************************************
% * Anexos
% **************************************************
\appendix
\section{Anexos}

\end{document}