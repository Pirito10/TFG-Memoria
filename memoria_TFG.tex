\documentclass[%
    paper=A4,               % Tamaño de papel A4.
    twoside,                % Páxinas pares e impares con deseño diferenciado.
    parskip=full,           % Renglón libre entre parágrafos.
    10pt,                   % Tamaño de fonte. Non inferior a 10 puntos. 
    titlepage=on,           % Páxina independente para o título
    captions=tableabove,    % Os títulos das táboas na parte superior
]{article}                  % Baseado no tipo "article" 

% **************************************************
% * Idioma: galician, spanish, english
% **************************************************
\usepackage[spanish]{babel}

\usepackage{tfgteleco}
\usepackage{lipsum}


% **************************************************
% * Datos para a portada do TFG
% **************************************************
\title{Diseño y desarrollo de una funcionalidad de búsqueda y filtrado de cursos en el catálogo de cursos en línea}
\author{Aarón Riveiro Vilar}
\titores{Manuel Caeiro Rodríguez}
\curso{2025/2026}

% **************************************************
% * Cabeceiras (opcionais). Actívanse con "true" 
% **************************************************
\cabeceiras{true}

% **************************************************
% * Inicio do documento
% **************************************************
\begin{document}

\maketitle
% **************************************************
% * Táboa de contidos (Opcional)
% **************************************************
\tableofcontents
\clearpage

% **************************************************
% * Inicio de contidos
% **************************************************

\section{Introdución}
O documento do TFG debe estar estruturado como un informe de non máis de 20 páxinas (incluíndo táboas e gráficas e a portada establecida) seguido de tantos apéndices coma sexa necesario para que a memoria final sexa o máis autocontida posible.  

As entradas da bibliografía compoñeranse utilizando o utilizando o estilo de citas do IEEE \cite{goossens1994latex} ou o estilo de citas Harvard. Cada entrada debe conter información suficiente para localizar directamente os autores do documento, o título, a publicación que o contén e o ano de publicación. No caso do documentos electrónicos, incluirase unha URL válida para o acceso ao mesmo e a data de consulta do documento.

Introdución ao problema.

\subsection{Obxectivos}
Descripción dos obxectivos do TFG.

A Figura \ref{fig:exemplo} é un exemplo de como incorporar unha figura.

\begin{figure}[htb!]
    \centering
    \includegraphics[width=0.5\textwidth]{images/logo_uvigo.pdf}
    \caption{Figura de exemplo}
    \label{fig:exemplo}
\end{figure}

\subsection{Metodoloxía}
Metodoloxía seguida

% TEXTO DE RECHEO PARA EXEMPLO: Borrar
\textcolor{gray}{ \newline \it
\lipsum[1-2]
}

\section{Situación actual}
Análise de necesidades e estudio do estado da arte.

O Cadro \ref{table:exemplo} mostra un exemplo de táboa (existen moitos máis estilos e contornos).

\begin{table}[h!]
\caption{Táboa de exemplo}
\label{table:exemplo}
\centering
\begin{tabular}{ |p{6cm}|p{4cm}|p{4cm}|  }
\hline
\multicolumn{3}{|c|}{Country List} \\
\hline
Country Name or Area Name& ISO ALPHA 2 Code &ISO ALPHA 3 \\
\hline
Afghanistan & AF &AFG \\
Aland Islands & AX   & ALA \\
Albania &AL & ALB \\
Algeria    &DZ & DZA \\
American Samoa & AS & ASM \\
Andorra & AD & AND   \\
Angola & AO & AGO \\
\hline
\end{tabular}
\end{table}

\section{Deseño}
Decisións técnicas tomadas, incluíndo o uso de estándares ou normativas ou xustificando a súa ausencia.

O Listado \ref{listado:exemplo} mostra un exemplo de código Python.

\begin{lstlisting}[language=Python, caption=Exemplo en Python, label=listado:exemplo]
import numpy as np
    
def incmatrix(genl1,genl2):
    m = len(genl1)
    n = len(genl2)
    M = None #to become the incidence matrix
    VT = np.zeros((n*m,1), int)  #dummy variable
    
    #compute the bitwise xor matrix
    M1 = bitxormatrix(genl1)
    M2 = np.triu(bitxormatrix(genl2),1) 

    for i in range(m-1):
        for j in range(i+1, m):
            [r,c] = np.where(M2 == M1[i,j])
            for k in range(len(r)):
                VT[(i)*n + r[k]] = 1;
                VT[(i)*n + c[k]] = 1;
                VT[(j)*n + r[k]] = 1;
                VT[(j)*n + c[k]] = 1;
                
                if M is None:
                    M = np.copy(VT)
                else:
                    M = np.concatenate((M, VT), 1)
                
                VT = np.zeros((n*m,1), int)
    
    return M
\end{lstlisting}

\section{Resultados}
Resultados acadados, xustificando os motivos polos que non se poido lograr a consecución de tódolos obxectivos (de ser o caso).

% TEXTO DE RECHEO PARA EXEMPLO: Borrar
\textcolor{gray}{ \newline \it
\lipsum[6]
}

\section{Conclusións}
Discusión e conclusións. Valorando, de ser o caso, o impacto en aspectos de responsabilidade legal, ética e profesional relacionados co TFG (por exemplo, aspectos relacionados coa privacidade, a seguridade, etc.).

% TEXTO DE RECHEO PARA EXEMPLO: Borrar
\textcolor{gray}{ \newline \it
\lipsum[7]
}

% **************************************************
% * Bibliografía
% **************************************************
% * Débese completar no ficheiro "references.bib" 
% * incluíndo alí as entradas en formato BibTeX. 
% * As referencias en BibTeX pódense extraer 
% * por exemplo de Google Scholar
% **************************************************
\bibliographystyle{IEEEtran} 
\bibliography{references}

\clearpage
\appendix
\section{Anexos técnicos}
Anexos técnicos que incorporen información adicional para calquera dos epígrafes anteriores ou incorporando aspectos novos que teñan relevancia para o traballo realizado (por exemplo, apartados propios das normas UNE 157001:2014 e UNE-ISO 21500:2013, se procede).

\end{document}
