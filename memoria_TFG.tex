% **************************************************
% * Opciones de clase y globales
% **************************************************
\documentclass[%
    paper=A4,               % Tamaño de papel A4
    twoside,                % Páginas pares e impares con diseño diferenciado
    parskip=full,           % Renglón libre entre párrafos
    10pt,                   % Tamaño de fuente. No inferior a 10 puntos.
    titlepage=on,           % Página independiente para el título
    captions=tableabove,    % Los títulos de las tablas en la parte superior
]{article}                  % Basado en el tipo "article" 

% **************************************************
% * Idioma y paquete de formato
% **************************************************
\usepackage[spanish]{babel}
\usepackage{tfgteleco}

% **************************************************
% * Datos para la portada del TFG
% **************************************************
\title{Diseño y desarrollo de una funcionalidad de búsqueda y filtrado de cursos en el catálogo de cursos en línea}
\author{Aarón Riveiro Vilar}
\titores{Manuel Caeiro Rodríguez}
\curso{2025/2026}

% **************************************************
% * Cabeceras (opcionales)
% **************************************************
\cabeceiras{true}

% **************************************************
% * Inicio del documento
% **************************************************
\begin{document}

\maketitle
% **************************************************
% * Tábla de contenidos (opcional)
% **************************************************
\tableofcontents
\clearpage

% **************************************************
% * Inicio de contenidos
% **************************************************

\section{Introducción}
En los últimos años, el crecimiento sostenido de los sistemas de información digitales ha dado lugar a repositorios con volúmenes de datos cada vez mayores y más heterogéneos. En este contexto, la mera disponibilidad de la información ya no resulta suficiente: es necesario dotar a los sistemas de mecanismos eficaces que permitan localizar, filtrar y presentar los datos de forma relevante para el usuario. La disciplina de la recuperación de información (Information Retrieval) estudia precisamente los métodos y modelos que permiten acceder de manera eficiente a grandes colecciones de documentos, maximizando la relevancia de los resultados ofrecidos frente a una consulta determinada \cite{information_retrieval}.

A medida que aumenta el tamaño y la complejidad de los repositorios, la experiencia de usuario pasa a depender en gran medida de la calidad del sistema de búsqueda y filtrado. No solo es importante la rapidez de respuesta, sino también la precisión y exhaustividad de los resultados, así como la capacidad del sistema para interpretar la intención de búsqueda del usuario y ofrecer mecanismos de refinamiento progresivo mediante filtros, facetas o criterios adicionales \cite{retrieval_accuracy}. Un sistema de búsqueda limitado a coincidencias literales sobre un único campo de texto puede resultar insuficiente en entornos donde los datos presentan múltiples dimensiones y atributos relevantes.

Por ello, el diseño e implementación de estrategias avanzadas de búsqueda y filtrado constituye un elemento clave en cualquier sistema de información orientado al usuario final, especialmente cuando se persigue facilitar el acceso eficiente a contenidos estructurados y mejorar la usabilidad global de la plataforma.

\subsection{Objetivos}
Descripción dos obxectivos do TFG.

A Figura \ref{fig:exemplo} é un exemplo de como incorporar unha figura.

\begin{figure}[htb!]
    \centering
    \includegraphics[width=0.5\textwidth]{images/logo_uvigo.pdf}
    \caption{Figura de exemplo}
    \label{fig:exemplo}
\end{figure}

\subsection{Metodoloxía}
Metodoloxía seguida

\section{Situación actual}
Análise de necesidades e estudio do estado da arte.

O Cadro \ref{table:exemplo} mostra un exemplo de táboa (existen moitos máis estilos e contornos).

\begin{table}[h!]
    \caption{Tabla de ejemplo}
    \label{table:exemplo}
    \centering
    \begin{tabular}{ |p{6cm}|p{4cm}|p{4cm}|  }
        \hline
        \multicolumn{3}{|c|}{Country List}                         \\
        \hline
        Country Name or Area Name & ISO ALPHA 2 Code & ISO ALPHA 3 \\
        \hline
        Afghanistan               & AF               & AFG         \\
        Aland Islands             & AX               & ALA         \\
        Albania                   & AL               & ALB         \\
        Algeria                   & DZ               & DZA         \\
        American Samoa            & AS               & ASM         \\
        Andorra                   & AD               & AND         \\
        Angola                    & AO               & AGO         \\
        \hline
    \end{tabular}
\end{table}

\section{Diseño}
Decisións técnicas tomadas, incluíndo o uso de estándares ou normativas ou xustificando a súa ausencia.

O Listado \ref{listado:exemplo} mostra un ejemplo de código Python.

\begin{lstlisting}[language=Python, caption=Ejemplo en Python, label=listado:exemplo]
import numpy as np
    
def incmatrix(genl1,genl2):
    m = len(genl1)
    n = len(genl2)
    M = None #to become the incidence matrix
    VT = np.zeros((n*m,1), int)  #dummy variable
    
    #compute the bitwise xor matrix
    M1 = bitxormatrix(genl1)
    M2 = np.triu(bitxormatrix(genl2),1) 

    for i in range(m-1):
        for j in range(i+1, m):
            [r,c] = np.where(M2 == M1[i,j])
            for k in range(len(r)):
                VT[(i)*n + r[k]] = 1;
                VT[(i)*n + c[k]] = 1;
                VT[(j)*n + r[k]] = 1;
                VT[(j)*n + c[k]] = 1;
                
                if M is None:
                    M = np.copy(VT)
                else:
                    M = np.concatenate((M, VT), 1)
                
                VT = np.zeros((n*m,1), int)
    
    return M
\end{lstlisting}

\section{Resultados}
Resultados acadados, xustificando os motivos polos que non se poido lograr a consecución de tódolos obxectivos (de ser o caso).

\section{Conclusiones}
Discusión e conclusións. Valorando, de ser o caso, o impacto en aspectos de responsabilidade legal, ética e profesional relacionados co TFG (por exemplo, aspectos relacionados coa privacidade, a seguridade, etc.).

% **************************************************
% * Bibliografía
% **************************************************
% * Débese completar no ficheiro "references.bib" 
% * incluíndo alí as entradas en formato BibTeX. 
% * As referencias en BibTeX pódense extraer 
% * por exemplo de Google Scholar
% **************************************************
\bibliographystyle{IEEEtran}
\bibliography{references}

\clearpage
\appendix
\section{Anexos técnicos}
Anexos técnicos que incorporen información adicional para calquera dos epígrafes anteriores ou incorporando aspectos novos que teñan relevancia para o traballo realizado (por exemplo, apartados propios das normas UNE 157001:2014 e UNE-ISO 21500:2013, se procede).

\end{document}